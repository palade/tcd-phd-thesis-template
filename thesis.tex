%%%%%%%%%%%%%%%%%%%%%%%%%%%%%%%%%%%%%%%%%%%%%%%%%%%%%%%%%%%%%%%%%%%%%%%%%%%%%
%%%
%%% File: utthesis2.doc, version 2.0jab, February 2002
%%%
%%% Based on: utthesis.doc, version 2.0, January 1995
%%% =============================================
%%% Copyright (c) 1995 by Dinesh Das.  All rights reserved.
%%% This file is free and can be modified or distributed as long as
%%% you meet the following conditions:
%%%
%%% (1) This copyright notice is kept intact on all modified copies.
%%% (2) If you modify this file, you MUST NOT use the original file name.
%%%
%%% This file contains a template that can be used with the package
%%% utthesis.sty and LaTeX2e to produce a thesis that meets the requirements
%%% of the Graduate School of The University of Texas at Austin.
%%%
%%% All of the commands defined by utthesis.sty have default values (see
%%% the file utthesis.sty for these values).  Thus, theoretically, you
%%% don't need to define values for any of them; you can run this file
%%% through LaTeX2e and produce an acceptable thesis, without any text.
%%% However, you probably want to set at least some of the macros (like
%%% \thesisauthor).  In that case, replace "..." with appropriate values,
%%% and uncomment the line (by removing the leading %'s).
%%%
%%%%%%%%%%%%%%%%%%%%%%%%%%%%%%%%%%%%%%%%%%%%%%%%%%%%%%%%%%%%%%%%%%%%%%%%%%%%%

\documentclass[a4paper, 11pt, oneside]{report}         %% LaTeX2e document.
\usepackage {tcdthesis}              %% Preamble.

%%%%%%%%%%%%%%%%%%%%%%%%%%%%%%%
%MYPACKAGES
\usepackage{graphicx}
%\usepackage{hyperref}
%\usepackage[disable]{todonotes} !!!! UNCOMMENT THIS BEFORE SENDING FOR REVIEW
\usepackage{todonotes}

% This is for the line in top of chapter
\usepackage{fancyhdr}

\usepackage{amsmath}
\usepackage{algorithm,algorithmic}

% Load the package
\usepackage[acronym,style=super,nonumberlist,toc]{glossaries}

% This is to break long citations
\usepackage{breakcites}
 
% Generate the glossary
\makenoidxglossaries


%%%%%%%%%%%%%%%%%%%%%%%%%%%%%%%

%\mastersthesis                     %% Uncomment one of these; if you don't
\phdthesis                         %% use either, the default is \phdthesis.

%\thesisdraft                      %% Uncomment this if you want a draft
                                     %% version; this will print a timestamp
                                     %% on each page of your thesis.

\leftchapter                       %% Uncomment one of these if you want
%\centerchapter                      %% left-justified, centered or
%\rightchapter                       %% right-justified chapter headings.
                                     %% Chapter headings includes the
                                     %% Contents, Acknowledgments, Lists
                                     %% of Tables and Figures and the Vita.
                                     %% The default is \centerchapter.

% \singlespace                       %% Uncomment one of these if you want
\oneandhalfspace                   %% single-spacing, space-and-a-half
% \doublespace                       %% or double-spacing; the default is
                                     %% \oneandhalfspace, which is the
                                     %% minimum spacing accepted by the
                                     %% Graduate School.

\renewcommand{\thesisauthor}{Bart Simpson}  %% Your official UT name.
\renewcommand{\thesismonth}{October}       %% Your month of graduation.
\renewcommand{\thesisyear}{2018}             %% Your year of graduation.
\renewcommand{\thesistitle}{This is a very long title}      %% The title of your thesis; use mixed-case.
\renewcommand{\thesisauthorpreviousdegrees}{}  %% Your previous degrees, abbreviated; separate multiple degrees by commas.
\renewcommand{\thesissupervisor}{Homer Simpson}     %% Your thesis supervisor; use mixed-case and don't use any titles or degrees.
% \renewcommand{\thesiscosupervisor}{}                %% Your PhD. thesis co-supervisor; if any.

% \renewcommand{\thesiscommitteemembera}{}
% \renewcommand{\thesiscommitteememberb}{}
% \renewcommand{\thesiscommitteememberc}{}
% \renewcommand{\thesiscommitteememberd}{}
% \renewcommand{\thesiscommitteemembere}{}
% \renewcommand{\thesiscommitteememberf}{}
% \renewcommand{\thesiscommitteememberg}{}
% \renewcommand{\thesiscommitteememberh}{}
% \renewcommand{\thesiscommitteememberi}{}

\renewcommand{\thesisauthoraddress}{Dublin, Ireland}
\renewcommand{\thesisdedication}{}     %% Your dedication, if you have one; use "\\" for linebreaks.
\renewcommand{\baselinestretch}{1.5}

%%%%%%%%%%%%%%%%%%%%%%%%%%%%%%%%%%%%%%%%%%%%%%%%%%%%%%%%%%%%%%%%%%%%%%%%%%%%%
%%%
%%% The following commands are all optional, but useful if your requirements
%%% are different from the default values in utthesis.sty.  To use them,
%%% simply uncomment (remove the leading %) the line(s).

% \renewcommand{\thesiscommitteesize}{...}
                                     %% Uncomment this only if your thesis
                                     %% committee does NOT have 5 members
                                     %% for \phdthesis or 2 for \mastersthesis.
                                     %% Replace the "..." with the correct
                                     %% number of members.

% \renewcommand{\thesisdegree}{...}  %% Uncomment this only if your thesis
                                     %% degree is NOT "DOCTOR OF PHILOSOPHY"
                                     %% for \phdthesis or "MASTER OF ARTS"
                                     %% for \mastersthesis.  Provide the
                                     %% correct FULL OFFICIAL name of
                                     %% the degree.

% \renewcommand{\thesisdegreeabbreviation}{...}
                                     %% Use this if you also use the above
                                     %% command; provide the OFFICIAL
                                     %% abbreviation of your thesis degree.

% \renewcommand{\thesistype}{...}    %% Use this ONLY if your thesis type
                                     %% is NOT "Dissertation" for \phdthesis
                                     %% or "Thesis" for \mastersthesis.
                                     %% Provide the OFFICIAL type of the
                                     %% thesis; use mixed-case.

% \renewcommand{\thesistypist}{...}  %% Use this to specify the name of
                                     %% the thesis typist if it is anything
                                     %% other than "the author".

%%%%%%%%%%%%%%%%%%%%%%%%%%%%%%%%%%%%%%%%%%%%%%%%%%%%%%%%%%%%%%%%%%%%%%%%%%%%%%%%

\usepackage{subcaption}

\pagestyle{fancy}
\fancyhf{}
%\fancyhead[EL]{\nouppercase\leftmark}
\fancyhead[LO]{\itshape\nouppercase{\leftmark}}
\fancyhead[RO,LE]{\thepage}

% Separate file to fine all the acronyms
% This is the acronym list.

% This is how you use it in the main body 
% \acrshort{soa} This form is used for the short form of the acronym
% \acrfull{soa} This form is used for the long form of the acronym


\newacronym{ble}{BLE}{Bluetooth}
\newacronym{soa}{SOA}{Service Oriented Architecture}
\newacronym{soas}{SOAs}{Service Oriented Architectures}
\newacronym{vanet}{VANET}{Vehicular Ad-hoc Network}
\newacronym{vanets}{VANETs}{Vehicular Ad-hoc Networks}
\newacronym{wsn}{WSN}{Wireless Sensor Network}


% Reduces the distance between acronyms
\renewcommand*{\glsgroupskip}{}

% Bold font for acronyms
\renewcommand{\glsnamefont}[1]{\textbf{#1}}

\hyphenation{op-ti-mi-sa-tion}

\begin{document}                                  

\thesistitlepage                      %% Generate the title page.
\thesisdeclarationpage				  %% Generate the declaration page.
\thesispermissionpage				  %% Generate the copyright permission page
\thesisdedicationpage                 %% Generate the dedication page.


%% Use this to write your
%% acknowledgments; it can be anything
%% allowed in LaTeX2e par-mode.

\begin{thesisacknowledgments}      

There are a number of people without whom this thesis would not have been possible, and to whom I am greatly indebted and dedicate this thesis....


\end{thesisacknowledgments}                       



\include{abstract}



\tableofcontents                                  %% Generate table of contents.
\listoftables                                     %% Uncomment this to generate list of tables.
\listoffigures                                    %% Uncomment this to generate list of figures.


\printnoidxglossary[type=\acronymtype,title={List of Abbreviations}]

% UNCOMMENT THIS !!!!!!!!!!!!!!!
%\begin{relatedpublications}

This thesis is based on the following peer-reviewed publications:

\begin{itemize}\fontsize{11pt}{11pt}\selectfont
	
	\item \textbf{My publication 1}
				
\end{itemize}



\end{relatedpublications}

% COMMENT THIS !!!!!!!!!!!!!!!
%\todototoc
% COMMENT THIS !!!!!!!!!!!!!!!
%\listoftodos


%% Include thesis chapters here
\chapter{Introduction}\label{chap:introduction}

Lorem ipsum dolor sit amet, consectetur adipiscing elit. Nulla nec neque scelerisque, bibendum mauris vel, commodo neque. Etiam nec erat diam. Etiam sed aliquet orci, eu sollicitudin orci. Vestibulum ante ipsum primis in faucibus orci luctus et ultrices posuere cubilia Curae; Donec vel augue sodales, consectetur tortor vitae, rutrum eros. In hac habitasse platea dictumst. Cras suscipit nisl vel urna cursus hendrerit. Sed accumsan ipsum eget nisi volutpat, vel commodo urna ornare. Nunc sit amet turpis neque. Proin eleifend massa et enim tristique efficitur. Maecenas rutrum sapien ac dolor tempor volutpat. Vivamus at erat in diam ultricies ullamcorper. Proin quam lectus, elementum at lorem at, aliquet commodo quam. Aliquam fermentum eros dictum laoreet ultrices~\cite{smith2011lorem}.  %introduction    			                      
\chapter{Background}\label{chap:relatedwork} %state of the art                              
\chapter{Design}\label{chap:design} %design							
\chapter{Implementation}\label{chap:implementation} %implementation			
\chapter{Evaluation}\label{chap:evaluation} %evaluation					
\chapter{Conclusion}\label{chap:conclusion} %conclusion					


% UNCOMMENT THIS IF YOU HAVE APPENDICES !!!
%\addcontentsline {toc}{chapter}{Appendices}       %% Force Appendices to appear in contents
%\begin{appendix}
%\include{appendix1}
% include{appendix2}
%\end{appendix}


%\addcontentsline {toc}{chapter}{Bibliography}     %% Force Bibliography to appear in contents
%\bibliographystyle{ieeetr}  %% Start your bibliography here; you can

\bibliographystyle{apalike}  		%<---REMOVE COMMENT
\bibliography{ref}          			%<---REMOVE COMMENT
%% also use the \bibliography command to generate your bibliography.


\end{document} 
